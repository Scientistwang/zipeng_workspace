\documentclass[17pt]{beamer}
\usetheme{Dresden}
\usepackage[utf8]{inputenc}
\usepackage{amsmath}
\usepackage{amsfonts}
\usepackage{amssymb}
\author{Zipeng Wang}
\title{Bifurcation analysis of microbiome steady states}
%\setbeamercovered{transparent} 
%\setbeamertemplate{navigation symbols}{} 
%\logo{} 
%\institute{} 
%\date{} 
%\subject{} 
\begin{document}

\begin{frame}
\titlepage
\end{frame}

%\begin{frame}
%\tableofcontents
%\end{frame}

\begin{frame}{1}

\end{frame}

\end{document}% Use only LaTeX2e, calling the article.cls class and 12-point type.

\documentclass[12pt]{article}

% Users of the {thebibliography} environment or BibTeX should use the
% scicite.sty package, downloadable from *Science* at
% www.sciencemag.org/about/authors/prep/TeX_help/ .
% This package should properly format in-text
% reference calls and reference-list numbers.

\usepackage{graphicx}

% Use times if you have the font installed; otherwise, comment out the
% following line.

\usepackage{times}

% The preamble here sets up a lot of new/revised commands and
% environments.  It's annoying, but please do *not* try to strip these
% out into a separate .sty file (which could lead to the loss of some
% information when we convert the file to other formats).  Instead, keep
% them in the preamble of your main LaTeX source file.


% The following parameters seem to provide a reasonable page setup.

\topmargin 0.0cm
\oddsidemargin 0.2cm
\textwidth 16cm 
\textheight 21cm
\footskip 1.0cm


%The next command sets up an environment for the abstract to your paper.

\newenvironment{sciabstract}{%
\begin{quote} \bf}
{\end{quote}}


% If your reference list includes text notes as well as references,
% include the following line; otherwise, comment it out.

\renewcommand\refname{References and Notes}

% The following lines set up an environment for the last note in the
% reference list, which commonly includes acknowledgments of funding,
% help, etc.  It's intended for users of BibTeX or the {thebibliography}
% environment.  Users who are hand-coding their references at the end
% using a list environment such as {enumerate} can simply add another
% item at the end, and it will be numbered automatically.

\newcounter{lastnote}
\newenvironment{scilastnote}{%
\setcounter{lastnote}{\value{enumiv}}%
\addtocounter{lastnote}{+1}%
\begin{list}%
{\arabic{lastnote}.}
{\setlength{\leftmargin}{.22in}}
{\setlength{\labelsep}{.5em}}}
{\end{list}}


% Include your paper's title here

\title{Steady States in 2D gLV Equations} 


% Place the author information here.  Please hand-code the contact
% information and notecalls; do *not* use \footnote commands.  Let the
% author contact information appear immediately below the author names
% as shown.  We would also prefer that you don't change the type-size
% settings shown here.

\author
{Zipeng Wang
}

% Include the date command, but leave its argument blank.

\date{}



%%%%%%%%%%%%%%%%% END OF PREAMBLE %%%%%%%%%%%%%%%%



\begin{document} 

% Double-space the manuscript.

\baselineskip24pt

% Make the title.

\maketitle 

% Place your abstract within the special {sciabstract} environment
Nondimensionalized 2D gLV equations are
\begin{eqnarray}
\frac{dx_{a}}{dt} &= x_a(\mu_a-x_a+M_{ab}x_b)\\
\frac{dx_{b}}{dt} &= x_b(\mu_b-x_b+M_{ba}x_a)
\end{eqnarray}

and has three steady states at $(1,0)$,$(0,\mu_b)$, and $(\frac{1-M_{ab}\mu_b}{1-M_{ab}M_{ba}},\frac{\mu_b-M_{ba}}{1-M_{ab}M_{ba}})$. Taking $\mu_b = 1$, the following graph summarizes the stability of these three steady states.
\begin{figure}[ht]
	\centering
	\includegraphics[width = 4.5in ]{figure_1}% Here is how to import EPS art
	\caption{\label{fig:epsart} Four graphs of steady states are shown above, including their respective parameter ranges. Black dots shows stable steady state, and hollowed dots shows unstable steady states.  }
\end{figure}

The graph in the middle shows the limiting case where $M_{ab}=M_{ba}=1$. The cross means both eigenvalues are 0, and gray dots means that one eigenvalue is 0 and the other is negative.









\bibliography{scibib}

\bibliographystyle{Science}






\end{document}




















